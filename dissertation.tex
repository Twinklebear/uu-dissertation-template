%%% -*-LaTeX-*-
%%% ====================================================================
%%% This is a sample top-level LaTeX-2e file for typesetting a thesis
%%% or dissertation at the University of Utah.  Most students find it
%%% convenient to start with a COPY of this file as a template, and
%%% then alter that copy to match their needs.
%%%
%%% There is an associated Unix Makefile that can be similarly
%%% customized, and then the only command ever needed to typeset the
%%% complete thesis is the single word "make".  Of course, during
%%% writing and typesetting, not all of the steps are needed, so
%%% often, one can just name a convenience target such as "make
%%% dvi-pass" or "make pdf-pass" to do just a single pass of LaTeX and
%%% BibTeX.
%%%
%%% There should be no, or very few, macro definitions in this file;
%%% any needed belong in a private style file, called mythesis.sty,
%%% and input below after all other packages.  The bulk of this file
%%% should just be command invocations, and any arguments that they
%%% need.
%%%
%%% We exploit the fact that TeX ignores spaces after command names to
%%% line up arguments for better readability.
%%%
%%% Each chapter should be a separate complete file, so that you can
%%% insert a command like \includeonly{chap_intro} before the first
%%% \include{chap_xxx} command to avoid typesetting all but the
%%% chapter that you are currently working on, to save time.
%%%
%%% Remember that occupants of job positions change jobs from time to
%%% time: YOU are responsible for ensuring correct names of all humans
%%% mentioned in this file!
%%%
%%% [16-Mar-2016]
%%% ====================================================================

\documentclass[11pt,Chicago]{uuthesis2e}

%%% Undefine two macros from uuthesis2e.cls that conflict with
%%% definitions in amsthm.sty that fail to check for prior definitions!
%%% NB: The amsthm package refines the LaTeX theorem environment,
%%% and the uuthesis-color-headings wraps that definition, so the
%%% amsthm package must be read first!
\let \proof    = \relax
\let \endproof = \relax
\usepackage {amsthm}

%%% ====================================================================
%%% Choose an alternate font family for the document if the TeX default
%%% of Computer Modern is not wanted:

\usepackage{mathpazo}
%%% ====================================================================
%%% Some miscellaneous Utah- and student-specific settings:
%%%
%%% Chapter is one level, section and subsection are the next two levels.

\fourlevels

%%% ====================================================================
%%% The remaining packages are required by this particular thesis,
%%% but other theses will almost certainly need different packages:
%%%
%%% WARNING: MANY \LaTeX{} packages change dimensions, glue, and/or
%%% formatting styles, and such changes are likely to conflict with
%%% University of Utah Thesis Office requirements.  Therefore, minimize
%%% the number of packages that you include!

\usepackage {amsmath}
\usepackage {amssymb}
\usepackage {bm}
\usepackage {bibnames}
\usepackage {citesort}
\usepackage {graphicx}
\usepackage {graphpap}

%%% ====================================================================
% My own packages
\usepackage{url}
\usepackage[T1]{fontenc}
\usepackage[utf8]{inputenc}
\usepackage[dvipsnames]{xcolor}
\usepackage{microtype}
\usepackage{booktabs}
\usepackage{soul}
\usepackage{color}
\usepackage{bibentry}
\usepackage{cite}
\usepackage{makecell}
\usepackage{multirow}
\usepackage{tabularx}
\usepackage[plain]{algorithm}
\usepackage[noend]{algpseudocode}
\usepackage[capitalize]{cleveref}
\crefname{paragraph}{section}{sections}
\Crefname{paragraph}{Section}{Sections}
\usepackage{relsize}
\usepackage{float}

% Christine mentions for expedited service from thesis office,
% only include sections in the TOC. Left showing all for now
% so it's easier to see the structure
\setcounter{tocdepth}{3}

\nobibliography*
\usepackage{subcaption}
\captionsetup{compatibility=false,labelfont={bf,color=utahheadingcolor}}

\DeclareCaptionLabelFormat{figure}{Fig.~\thefigure}
\captionsetup[figure]{labelformat=figure}

\usepackage{minted}
\usemintedstyle{tango}

% Minted extra styling
\renewcommand\theFancyVerbLine{\small\arabic{FancyVerbLine}}
\definecolor{mintedbgcolor}{rgb}{0.95,0.95,0.95}
\usepackage{xcolor}
\usepackage{tcolorbox}
\usepackage{minted}
\setminted{fontsize=\small,mathescape}
\BeforeBeginEnvironment{minted}{
    \begin{tcolorbox}[colback=mintedbgcolor,left=1mm]
}
\AfterEndEnvironment{minted}{\end{tcolorbox}}

%%% ====================================================================
%%% Support for a subject index:

\usepackage {uuthesis-index}

%%% ====================================================================
%%% The various uuthesis-*.sty packages must come AFTER all other
%%% system-provided packages, so that they can correctly override
%%% settings from those packages.

%%% Include latest updates for 2016 (WARNING: the name is subject to
%%% change: see http://www.math.utah.edu/pub/uuthesis/ for the most
%%% current version.)

\usepackage {uuthesis-2016-h}  % MANDATORY package

%%% This is an OPTIONAL package that sets chapter and sectional headings
%%% in color:
%%% Use one or the other of these:
% \usepackage {color}
\definecolor{utahheadingcolor} {rgb}  {0.0, 0.0, 0.0}
\definecolor{utahtheoremcolor} {rgb}  {0.0,0.0,0.0} % purple4
\definecolor{utahtheoremcolor} {rgb}  {0.0,0.0,0.0} % brown4

%%% Here is another, and more convenient, way to define colors, via
%%% aliases of named colors in the X11-derived rgb.sty file

\usepackage{coloralias}
\definecoloralias{utahheadingcolor}{black}
\definecoloralias{utahtheoremcolor}{black}

%%% ====================================================================
%%% Support for a subject index:

\usepackage {uuthesis-index}

%%% ====================================================================
%%% This single user-specific file is where all personal customizations
%%% and macro definitions should be placed, and it should come LAST,
%%% after ALL OTHER packages, in case it needs to override some of their
%%% definitions.

\usepackage {mythesis}
\usepackage{marginnote}

\renewcommand{\thealgorithm}{\arabic{chapter}.\arabic{algorithm}}
\renewcommand{\thelisting}{\arabic{chapter}.\arabic{listing}}

%%% ====================================================================
% Will: it seems like the package doesn't have the right geometry of the
% pages? So restyle it to be right
\usepackage[letterpaper,total={6in,9in},left=1.25in,right=1.25in,top=1in,bottom=1in,
footskip=0in]{geometry}


%%% ====================================================================
%%% The student-specific front matter fields are defined here:

\author                 {Your Name}

\title{Your Title}

\thesistype             {dissertation}

\dedication             {}

%%% Most students need just a short degree name, like this:
\degree                 {Doctor of Philosophy}
%%% However, multiline degrees are possible, and are done like this:
\degree                 {Doctor of Philosophy \\
                         in \\
                         X}

%%% College- and Department-level definitions:
\approvaldepartment     {}
\department             {}
\graduatedean           {}
\departmentchair        {}

%%% The graduate student's committee members:
\committeechair         {}
\firstreader            {}
\secondreader           {}
\thirdreader            {}
\fourthreader           {}
\chairtitle             {}

%%% NB: It is rare, but possible, for there to be two chairs, For
%%% example, one student had
%%%
%%% \committeechair{\mbox{\small Andrej Cherkaev and Andrejs Treibergs}}
%%%
%%% The \mbox{} ensures that line breaks cannot happen, and the \small
%%% is necessary to make the names fit on the Dissertation Approval form

%%% Dates that must be adjusted for each academic term, and be permitted
%%% according to the University of Utah Thesis Office:
\submitdate             {}
\copyrightyear          {}

%%% Dates on which committee members approved the thesis
%\chairdateapproved      {DD MM YYYY}
\chairdateapproved      {\;\;\;\;\;\;\;\;\;\;\;\;\;\;\;\;\;\;\;\;\;\;\;\;\;\;}
\firstdateapproved      {\;\;\;\;\;\;\;\;\;\;\;\;\;\;\;\;\;\;\;\;\;\;\;\;\;\;}
\seconddateapproved     {\;\;\;\;\;\;\;\;\;\;\;\;\;\;\;\;\;\;\;\;\;\;\;\;\;\;}
\thirddateapproved      {\;\;\;\;\;\;\;\;\;\;\;\;\;\;\;\;\;\;\;\;\;\;\;\;\;\;}
\fourthdateapproved     {\;\;\;\;\;\;\;\;\;\;\;\;\;\;\;\;\;\;\;\;\;\;\;\;\;\;}

%%% ====================================================================
%%% Typesetting begins here:

\begin{document}

%% Comment out items by inserting a percent % character
\frontmatterformat
\titlepage
\copyrightpage
\dissertationapproval
\setcounter {page}     {2}             % UofU Thesis Office demands abstract on p. iii: start one lower
\preface    {content/abstract} {Abstract}
\dedicationpage
\tableofcontents
\listoffigures
\listoftables
%
%%% Optional front matter page(s), made from source "notation.tex". If
%%% you don't need it, then comment out the \optionalfront command
%%% line!  The first argument is the (unnumbered) section header for
%%% the text supplied by the file input by the second argument; that
%%% file must NOT contain \chapter, \section, \subsection, \ldots{}
%%% sectioning commands.

%\optionalfront {Notation and Symbols} {\input{notation}}

%%% Uncomment this is you want the contents of acknowledge.tex typeset here.
%%% Note that both "Acknowledgement" and "Acknowledgment" are accepted
%%% spellings of that word.

% \preface{acknowledge}{Acknowledgements}

\maintext       % Start normal page numbering: parts and chapters follow.

\pagestyle{headings} % NEW for sample-thesis-6


\chapter{Example Introduction Chapter}

Here's a citation so bibtex doesn't complain~\cite{usher_interactive_2020}



\numberofappendices = 0   % Set 0 for none, else number of appendices.
%\numberofappendices = 1   % Set 0 for none, else number of appendices.
%\appendix       % Chapters, sections are now appendix style A, A.1, A.2, B, C, D, ...

%\include{content/appendix_terminology}
%\include{appb}
%\include{appc}

%%% The choice of bibliography style is a major decision, jointly made
%%% by you, your thesis advisor and the thesis editor. Common choices are
%%% one of the four standard BibTeX styles (abbrv, alpha, plain, and unsrt),
%%% or enhanced styles like acm, amsplain, siam, and hundreds of others
%%% available in TeX Live, and other Unix and Windows TeX distributions.
%%%
%%% Do NOT handcode your reference list, because you are unlikely to
%%% achieve consistency or conformance to the University of Utah Thesis Office
%%% requirements: let BibTeX do that tedious job for you!
%%%
%%% Remember that reference-list metadata in BibTeX files remains
%%% constant across journals and publishers, and is are often reused
%%% in other documents and shared with others, whereas formatted
%%% reference-list styles change: with BibTeX, you only need to record
%%% the metadata once.
%%%
%%% If you prefer named, rather than numeric or tagged citations, you
%%% may use styles such as authordate{1,2,3,4}, chicago, harvard, or
%%% natbib.  Be aware, however, that most of those require an
%%% additional \usepackage{} command to supply \LaTeX{} with
%%% definitions of commands that the style needs, and that there are
%%% usually several flavors of LaTeX citation commands beyond the
%%% standard \cite{} command that you need to understand before you
%%% can use them properly in your prose.

% Sorted or unsorted?
\bibliographystyle{IEEEtranS}

%%% This can also specify a comma-separated list (without embedded
%%% spaces) of *.bib files found by BibTeX in its BIBINPUTS search
%%% path.  The argument \jobname means the base name of the top-level
%%% LaTeX file, avoiding an unnecessary filename dependence here.
%%%
%%% BibTeX writes only one .bbl file, no matter how many *.bib files
%%% are listed here, using the name \jobname.bbl.
%%%
%%% LaTeX reads BibTeX's formatted reference list from the file
%%% \jobname.bbl.

\bibliography{dissertation}

%%% The last part of this sample thesis is two specialized indexes,
%%% and a general topic index.  If the companion Makefile is used to
%%% create the DVI or PDF file for this work, the topic index excludes
%%% the lengthy list of free software packages.  However, the biology
%%% names of the first index are included in the topic index.

%%% Switch from thesis double spacing to single spacing for the three
%%% indexes, as a matter of style (to match the reference list), and
%%% for compactness.

\singlespace

%%% Define several index cross references (there are many more such
%%% in chap1.tex, but the examples here give a useful summary of how
%%% they are made):

%\index{DCT|see{discrete cosine transform}}
%\index{DWT|see{discrete wavelet transform}}
%\index{Borel measure ($\mu$)}
%\index{mu@$\mu$ (mu)|see{Borel measure}}
%\index{Escherichia coli@\bioname{Escherichia coli}|see{E. coli}}
%\index{transform|seealso{Discrete DCT Transform}}
%\index{transform|seealso{Fast Fourier Transform}}

%\renewcommand {\bioname} [1] {\emph{#1}}   % redefine to suppress color and indexing in index
%\renewcommand {\fsfname} [1] {\texttt{#1}} % redefine to suppress color and indexing in index

%\renewcommand {\indexname} {Binomial Nomenclature Index}

%\input{\jobname-bioname.ind}

%\renewcommand {\indexname} {Free Software Index}

%\input{\jobname-fsfname.ind}

%\renewcommand {\indexname} {Topic Index}

%\overfullrule = 0pt % suppress visible warnings about overfull hboxes
%\printindex

\end {document}
